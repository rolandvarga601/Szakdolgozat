\documentclass[../documentation.tex]{subfiles}
 
\begin{document}
\section*{Abstract}
\angol{The aim of this paper is to demonstrate the human-robot collaboration via a robot demo application. This application is a chess player program which can be executed on the robot controller. This makes it possible to play chess against the robot.}

\angol{A KUKA iiwa robotic arm and a KRC4 (KUKA Robot Controller) was used for this purpose.}

\angol{The electric gripper that grabs the chess pieces is attached to the robot arm. The controller for the gripper was designed by the same manufacturer as the gripper itself. Digital (and for the further development analog) input and output modules were also installed to send commands to the gripper controller. These modules communicate with the KRC4 via EtherCAT.}

\angol{The robot program uses image processing methods to recognize the individual chess steps of the partner during the game. A HD webcamera is connected to the robot controller via an USB port to take the pictures (the webcamera driver had to be installed offline). The camera is attached to the side of the gripper, thus it moves with the robot arm.}

\angol{To calibrate the image processing methods a calibration table is used. It has the same cell size as the chessboard but it consists of 10x9 cells instead of 8x8 . The robot program can find the individual chess fields (and pieces) based on a picture of the calibration table (setup). The image processing part and the extended chess program together can recognize the steps of the white chess pieces.}

\angol{The chess program used for the project is mainly based on an already developed application which also has graphical user interface (GUI). To process the result of the image processing module (positions of white pieces) it was required to add some extra functions. The response step can be determined by different algorithms.}

\angol{The robotic arm can execute the defined chess moves generated by the algorithms. Although some special moves are still under development such as the captures, castling, en passant and promotion. After every move the robot arm goes to the picture capturing position and after the chess partner gives a feedback that a move has been taken the program grabs a frame from the camera and processes it.}

\angol{The program processes are cyclic, the steps:
\begin{enumerate}
	\item capturing an image of the current state
	\item image processing, finding the move that was made
	\item generating response
	\item executing the move with the robotic arm (and with the gripper)
	\item moving to the image capturing position
\end{enumerate}}

\angol{Altogether the chess player program currently is able to generate and execute chess moves after recognizing the move of the human partner. After some smaller extensions and modifications it will be possible to play a whole game against the robot.}




\end{document}