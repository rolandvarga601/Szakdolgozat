\documentclass[../documentation.tex]{subfiles}
 
\begin{document}
\section{Alkalmazáshoz szükséges műszaki feltételek elemzése}
\subsection{Projekt részletes leírása}
A projekt célja egy olyan robot demo hardveres és szoftveres kidolgozása, amely képes egy emberrel (továbbfejlesztés után akár egy másik robottal) lejátszani egy sakkjátszmát. A demo az ember-robot kollaboráció bemutatására szolgál, fontos szempont az interakció biztonságos megvalósítása mind az emberre, mind a környező tárgyakra tekintettel.\\

A megvalósításhoz a következő problémák megoldására van szükség:
\begin{enumerate}
	\item szükséges biztonsági funkciók beüzemelése,
	\item a bábuk helyzetének felismerése az egyes lépések előtt és után,
	\item a bábuk megfogása és mozgatása (ide tartozik a kalibráció és a referenciafelvétel),
	\item sakkalgoritmus beágyazása a programba,
	\item a sakkbábúk és a tábla megtervezése és megvalósítása,
	\item jelzés a robotkar számára, ha lépés történt.
\end{enumerate}
A felsorolt pontok a projekt során következőképpen kerültek kidolgozásra:
\begin{itemize}
	\item A bábuk helyzetének detektálása a projekt során QR-kód kereső és olvasó képfeldogozó eljárásokon alapul. A kamera a roboton kerül rögzítésre.
	\item A bábuk mozgatása egy elektromosan vezérelt, párhuzamos megfogó (\angol{gripper}) segítségével történik.
	\item Ahhoz hogy a bábuk megfogása egyszerű legyen, azonos magasságú és azonos módszerrel megfogható bábuk készülnek.
	\item A biztonsági funkciók főként az tengelyekben ébredő plusz nyomatékok monitorozására és biztonsági zónák definiálására épül.
	\item A tábla és a robotkar, illetve a megfogó (egy jól definiált pontja) és a robotkar relatív helyzetének kalibrálására a robotvezérlő szoftverben elérhető alapfunkciók kerültek felhasználásra.
	\item A sakklépés megtörténtétét a robotvezérlőhöz kötött külső gombbal tudja a felhasználó jelezni.
	\item A képek fogadása, feldolgozása és a sakkalgoritmus futtatása mind a robotvezérlőn történik.
	\item Mivel a robotvezérlőn Java alapú környezet fut magas szinten, így a képfeldolgozó és a sakkozó programok is ebben lettek implementálva.
\end{itemize}

\subsection{A megfogó kialakításának és vezérlésének bemutatása}

\subsection{QR-kód generálás és képfeldolgozás}



\end{document}