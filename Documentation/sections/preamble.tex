\documentclass[../documentation.tex]{subfiles}
 
\begin{document}

\section{Bevezetés}
\subsection{Célkitűzés}
A szakdolgozat célja megismertetni az olvasóval az Ipar 4.0 (az ipari fejlődés egyik legújabb és legmeghatározóbb trendje) fontos elemét, az ember-robot kollaborációt. A fókuszban egy gyakorlati megvalósítás, egy demo áll, viszont általános irányelvek és módszerek is taglalásra kerülnek. A szakdolgozat feltételezi, hogy az olvasó rendelkezik alapszintű ismeretekkel az Ipar 4.0 főbb elemeit illetően. A dolgozat fő szerepe az Ipar 4.0 egy kisebb részletének praktikus, szemléletes bemutatása, de mégis érdemes rendszerszemléletűen hozzágondolni a többi elemét is, mivel így kaphatunk csak teljes képet ennek funkciójáról.

\subsection{Áttekintés}
Első körben a 4. ipari forradalom főbb vonásai kerülnek bemutatásra, különös tekintettel az ember-robot együttműködésre (későbbiekben HRC - \foreignlanguage{british}{Human-Robot Collaboration}). Ezen belül az ember és robot együttműködésének különböző szintjei is taglalásra kerülnek azért, hogy szemléletesebbek legyenek az előnyök. Ezt követően a szakdolgozat magját jelentő demóhoz használt KUKA LBR iiwa 7 R800 robotkarhoz tartozó biztonságtechnikai fogalmakat és specifikációkat ismertetem, mivel ennek keretében beszélhetünk a HRC-vel kapcsolatos funkciókról.

A projekt több, különböző típusú részfeladatból áll. A szakirodalomkutatás utáni fejezetek ezen részfeladatok kidolgozását tartalmazzák.

A projekt eredményeit \aref{sec:results}. és \aref{sec:osszefoglalas}. fejezetek tartalmazzák.

\end{document}