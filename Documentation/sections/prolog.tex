\documentclass[../documentation.tex]{subfiles}
 
\begin{document}
\section*{Előszó}
Napjainkban egyre növekszik az igény arra, hogy a robotok az emberekkel kapcsolatba léphessenek, ne legyenek kerítésekkel elválasztva. Ilyen módon sokkal inkább ki tudják segíteni az embert, egyfajta technikai asszisztens szerepet láthatnak el. Olyan munkákat képesek elvégezni, amely az ember számára fárasztó, monoton vagy veszélyes lehet. 

A különféle robotokkal már nem csak ipari környezetben, hanem akár a gyógyászat területén is találkozhatunk. Ilyen robotok képesek például rehabilitációs tornákkal segíteni a betegeknek. Az ilyen mértékű együttműködésre tervezett robotok kialakításánál fontos szempont, hogy az ember biztonságban érezze magát mellettük. Ha egy robot tud vigyázni a körülötte lévőkre, mozgása gördülékeny, kiszámítható és jól alkalmazkodó, akkor sokkal könnyebb a bizalmat megadni neki.

A szakdolgozat célja az ember és a robot finom együttműködését bemutató robotalkalmazás kidolgozása. Ez egy társasjáték formájában valósul meg: lehetőség nyílik egy ipari robot ellen egy sakkjátszma lejátszására.


\end{document}