\documentclass[../documentation.tex]{subfiles}
 
\begin{document}
\section{Eredmények, továbbfejlesztési irányok}
A projekt főként 4 modulból áll (képkészítés, képfeldolgozás, sakkprogram és robotvezérlés), ezeket kellett összekötni ahhoz, hogy sakkozásra képessé lehessen tenni a robotkart \aref{sec:projectdescription} fejezetben bemutatott konstrukcióval. A projekt összesített értékelését FEJEZET HIVATKOZÁS!!!!!! fejezet tartalmazza. Ebben a fejezetben az egyes modulok korlátai, kiforrottsága, megbízhatósága és továbbfejleszthetősége kerül taglalásra.

\subsection{Képkészítés értékelése}
A robotvezérlő (KRC4) (és a rajta futó szoftver) kompatibilis a választott Logitech C270 HD webkamerával. A kamerát egyszerűen a robotvezérlőn található USB bemenetek valamelyikére lehet kötni. A robotkar és a robotvezérlő több méteres távolsága miatt USB hosszabító kábelre van szükség, 2 méteres passzív toldókábellel megfelelő képet szolgáltat a kamera.

A kamera kezelő osztály (CameraHandler.java) HD képes webkamera kezelésére lett kialakítva. Minden bizonnyal egyéb képkészítésre alkalmas eszközt is tudna kezelni, de ez további teszteket igényelne. A kamera megnyitási ideje erősen változó lehet a tipus függvényében: a Logitech kamera esetén ez pár másodperc, míg a Microsoft kameránál fél vagy 1 perc is lehetett. A Windows alapfunkciójaként elérhető kamera alkalmazás mindkettő kamerát 1-2 másodperc alatt megnyitotta, szóval az OpenCV kamerakezelésében lehet különbség.

Továbbfejleszthető ez a modul:
\begin{itemize}
	\item a felbontás és képarány automatikus választásával,
	\item több csatlakoztatott kamera esetén a megfelelő kiválasztásával,
	\item illetve a torzítások ellensúlyozásával.
\end{itemize}

\subsection{Képfeldolgozás megbízhatósága}
A képfeldolgozás alapvetően 2 elemből áll: a mezőkalibrációból és a fehér bábuk kereséséből.

A mezőkalibráció főként a kalibrációs sakktáblamintán található sarokpontok kereséséből áll. Ezek alapján történik a mezőkről (pontosabban a bábuk tetejéről) készült képek kivágása. Az OpenCV 2.4.13.6-os verziónál a 3.4.4 megbízhatóbban működik, ugyanazon a képen megtalál olyan sarokpontokat, amiket a régebbi verzó nem. A sarokpontok megtalálása nem csak a kalibrációs tábláról alkotott képrészlet minőségétól függ, hanem a képen található egyéb mintázatoktól is, főként ha annak vonalai kontrasztosak. A kalibrációs képet utólagosan lehet módosítani, hogy a sakktáblán kívüli részt adott színnel ki legyen töltve. Ezt manuálisan kell megtenni, ami a jelen konstrukcióban elfogadható, de a továbbfejlesztési irányok (FEJEZET HIVATKOZÁS!!!) fejezetben ennél egyszerűbb alkalmazást biztosító módszerről is található leírás.

A fehér bábuk keresése a mezőkről kapott képeken belüli színszűrésen alapszik. A módszer RGB színskálát használ a zöld színű részek megtalálására, ami átlagosnak mondható fényviszonyok között kifejezetten jól működik. Rosszabb fényviszonyok között nem lett tesztelve a módszer, de ilyen esetben a HSV színskála használata (az RGB helyett) előnyösebb lehet. A tesztelt fényviszonyokkal a képek körülbelül 50\%-át tölti ki zöld szín, ha ott fehér bábu található (a tetejükre zöld lap van ragasztva). Ez alapján a bábuk pozíciójának kiértékelése egyszerű.

Továbbfejleszthetési irányok:
\begin{itemize}
	\item A projekt kidolgozásának idején az OpenCV egy újabb verzióját is kiadták, érdemes lehet azt beépíteni a programba.
	\item A kamerapozíciónálás során a megtalált sarokpontok szinekkel megjelölése (kvázi valós időben) elősegítené a megfelelő pozíció megtalálását.
	\item Az \ref{qrsection} fejezetben ismertetett QR kód alapú eljárás kidolgozása további funkciókat tenne elérhetővé.
\end{itemize}

\subsection{A sakkprogram értékelése}
A sakkprogram elsődleges forrása egy kész, Java alapű, felhasználói interfésszel rendelkező program volt, ebből lettek a projekthez szükséges elemek kiemelve. További függvényekkel kellett kiegészíteni ezt a programorészletet ahhoz, hogy a képfeldolgozó és a robotvezérlési modulokkal kompatibilis legyen.

Az sakkprogram magja fel van készítve arra, hogy egy teljes sakkjátszámát le lehessen játszani, viszont a képfeldolgozási résszel összekötés olyan módon lett implementálva, amely a sáncolást nem tudja ilyen formában felismerni és kezelni. Ez a probléma pár függvénykibővítéssel megoldható.

A sakkprogram által használt algoritmus tetszőlegesen módosítható lenne játék közben is, de ennek kezelésére nem készült külön felhasználói bemenet. Jelenleg a sakkprogram kódjának megváltoztatásával lehet ezt állítani.

A program beépített virtuális sakkórát is használ, emellett a játékosok nevei is személyre szabhatóak (ezt a nevet írja ki a program bármely játékoshoz kötődő esemény kiíratásánál). A sakkprogram magja képes lenne adott játékállásból folytatni egy játékot, illetve elmenteni az adott állást.

Ezzel a modullal kapcsolatos legjelentősebb fejlesztési irány a smartPAD-re kidolgozott felhasználói interfész kifejlesztése, így lehetőség nyílna játék közben a különböző paraméterek állítására, illetve a folyamatok monitorozására.

\subsection{A robotvezérlés értékelése}
\subsubsection{A megfogó irányítása}
A robotvezérlő EtherCAT kommunikációt használó, digitális és analóg I/O modulok segítségével tudja vezérelni a megfogót. A jelenleg implementált funkciókhoz elég csak a digitális modulok használata, de az analóg modulokra is szükség van az összes gripperfunkció kihasználáshoz (pl.: megfogó szorítóerejének beállítása, pozícióvisszacsatolás stb.).

A pozícióvisszacsatolás segítene eldönteni, hogy a megfogónak ténylegesen sikerült-e megfognia a bábut. A szorítóerő beállítása jelenleg hardveresen történik, de ezt a programból is lehetne vezérelni.

\subsubsection{A robotkar vezérlése}
A robotkar kalibrálása a sakktáblához gyorsan kivitelezhető. A robot képes a képkészítési pozíció ismételt felvételére, illetve a sakkalgoritmus által meghatározott lépések kivitelezésére, leszámítva azokat a lépéseket, amikhez több bábú mozgatása is szükséges (ütések, sáncolás, en passant). Ezen lépések implementálása nem ütközik hardveres problémába, így viszonylag könnyen megvalósíthatóak.

Továbbfejleszthető elemek:
\begin{itemize}
	\item Ha a robotkar ütközés hatására megáll, akkor jelenleg csak manuálisan lehet visszatéríteni az eredeti program folytatásához. Ezt a visszavezetést a programból is meg lehetne oldani, ez egy fejlesztési lehetőség.
	\item A robotkar jelenleg a robot bázis koordinátarendszerben mozog, de praktikusabb lenne a sakktáblához rendelt bázist használni. Ekkor a folyamatok a sakktábla különböző irányú dőléseire sokkal kevésbé lennének érzékenyek.
	\item A sakktábla bázispontjának felvétele ``handguiding'' (kézi vezetés) használatával gyorsabb és kényelmesebb lenne.
\end{itemize}






















\end{document}