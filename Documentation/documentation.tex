%------------------------------------------
% Szakdolgozat
% 
% 
% Varga Roland
%
% Konzulens: Magyar László
%------------------------------------------

\documentclass[12pt,a4paper,twoside]{article}
\usepackage{dissert}
% \renewcommand{\familydefault}{ptm}

\begin{document}
\pagenumbering{roman}
\author{Varga Roland}
\title{Ember és robot interakciójának demonstrációja\\
		Sakkozó iiwa robotkar}
%--Szennycímoldal-----------------------------------------
\newpage\thispagestyle{empty}
\begin{center}
     VARGA ROLAND\\
     SZAKDOLGOZAT
\end{center}
%--Sorozatcímoldal----------------------------------------
%\newpage\null\thispagestyle{empty}
\newpage\thispagestyle{empty}
\begin{center}
     BUDAPESTI MŰSZAKI ÉS GAZDASÁGTUDOMÁNYI EGYETEM\\
     GÉPÉSZMÉRNÖKI KAR\\
     MECHATRONIKA, OPTIKA ÉS GÉPÉSZETI INFORMATIKA TANSZÉK\\[1ex]
     \resizebox{2.5cm}{!}{
          \includegraphics{figures/mogi.png}
     }\\[1ex]
     SZAKDOLGOZATOK
\end{center}
%--Címoldal-----------------------------------------------
%\newpage\null\thispagestyle{empty}
\newpage\thispagestyle{empty}
\begin{titlepage} %environment for unique titlepage design
\centering
\resizebox{4.5cm}{!}{
  \includegraphics{figures/bme}
}\\[1ex]
{\bf BUDAPESTI MŰSZAKI ÉS GAZDASÁGTUDOMÁNYI EGYETEM}\\
{\bf GÉPÉSZMÉRNÖKI KAR}\\
{\bf MECHATRONIKA, OPTIKA ÉS GÉPÉSZETI INFORMATIKA TANSZÉK}\\[3cm]

{\LARGE \scshape Varga Roland}\\[2ex]
{\Large SZAKDOLGOZAT}\\[2ex]
{\Large \bf Ember és robot kooperációjának demonstrálása\\
		Sakkozó iiwa robotkar segítségével}\\[2ex]
{\itshape Demonstrating human-robot collaboration\\
				With chess-playing iiwa robotic arm}\\[5cm]

\begin{tabularx}{\textwidth}{XXXX}
Konzulens: & Témavezető: \\
\hspace{0.75cm} \itshape Magyar László & \hspace{0.75cm} \itshape Dr. Czmerk András \\
\hspace{0.75cm} tesztmérnök & \hspace{0.75cm} egyetemi adjunktus \\
\end{tabularx}\\[4cm]

{\Large Budapest, 2018}
\end{titlepage}

%--Záradék és nyilatkozatok-------------------------------
\documentclass[../documentation.tex]{subfiles}
 
\begin{document}
 \section*{Köszönetnyilvánítás}
 A szakdolgozat elkészítése során sokak segítségére számíthattam. Köszönöm a KUKA Hungária Kft.-nek, hogy eszközöket és színvonalas környezetet biztosított a projekt megvalósításához.Szeretném megköszönni a támogatást Magyar Lászlónak, konzulensemnek és munkatársamnak, akitől a legtöbb szakmai segítséget kaptam. Köszönet illeti a többi munkatársam is, akiket a szakdolgozat valamely eleme kapcsán felkerestem, mindegyikük rendkívül segítőkész volt.

Az általam tervezett sakkbábuk 3D kinyomtatásáért köszönettel tartozom a BME Polimertechnika Tanszéknek, ezen belül is Horváth Istvánnak és Dr. Kovács Norbertnek.

Köszönöm Dr. Czmerk Andrásnak, témavezetőmnek, hogy a szakdolgozat elkészítéséhez iránymutatást adott.

Ezentúl szeretném megköszönni a családomnak a belém vetett bizalmat, a törődést és a támogatást. Mindig számíthattam a segítségükre.

Köszönettel tartozom Pál Imrének, aki kiskoromban sakkozni tanított.

Végül de nem utolsó sorban szeretném megköszönni barátaimnak, hogy biztattak, és hogy segítettek a szakdolgozat és az egyetemi teendők mellett is jól érezni magam. Külön köszönettel tartozom a Martos Klubnak és a Drönk csapatának, hogy teret és lehetőséget biztosítottak a kikapcsolódáshoz.

\vspace{10mm}
\itshape{Budapest, 2018.12.05.}

\vspace{7mm}
\hspace{80mm} \itshape{Varga Roland}

\end{document}

%--Tartalomjegyzék----------------------------------------
\newpage\null\thispagestyle{empty}
\newpage\thispagestyle{plain}
\tableofcontents

%--Előszó-------------------------------------------------
%\documentclass[../documentation.tex]{subfiles}
 
\begin{document}
\section{Előszó}
Napjainkban egyre növekszik az igény arra, hogy a robotok az emberekkel kapcsolatba léphessenek, ne legyenek kerítésekkel elválasztva. Ilyen módon sokkal inkább ki tudják segíteni az embert, egyfajta technikai asszisztens szerepet láthatnak el. Olyan munkákat képesek elvégezni, amely az ember számára fárasztó, monoton vagy veszélyes lehet. 

A különféle robotokkal már nem csak ipari környezetben, hanem akár a gyógyászat területén is találkozhatunk. Ilyen robotok képesek például rehabilitációs tornákkal segíteni a betegeknek. Az ilyen mértékű együttműködésre tervezett robotok kialakításánál fontos szempont, hogy az ember biztonságban érezze magát mellettük. Ha egy robot tud vigyázni a körülötte lévőkre, mozgása gördülékeny, kiszámítható és jól alkalmazkodó, akkor sokkal könnyebb a bizalmat megadni neki.

A szakdolgozat célja az ember és a robot finom együttműködését bemutató robotalkalmazás kidolgozása. Ez egy társasjáték formájában valósul meg: lehetőség nyílik egy ipari robot ellen egy sakkjátszma lejátszására.


\end{document}
%--Jelölések jegyzéke-------------------------------------
\newpage\null\thispagestyle{empty}
\newpage\thispagestyle{plain}
\section*{Jelölések jegyzéke}

A táblázatban a többször előforduló jelölések magyar és angol nyelvű elnevezése, valamint a fizikai mennyiségek esetén annak mértékegysége található. Az egyes mennyiségek jelölése – ahol lehetséges – megegyezik hazai és a nemzetközi szak-irodalomban elfogadott jelölésekkel. A ritkán alkalmazott jelölések magyarázata első előfordulási helyüknél található.
%--Footer-------------------------------------------------
\newpage\null\thispagestyle{empty}
\newpage\thispagestyle{fancy}
%\include{boschfoot}

\pagenumbering{arabic}

%--Bevezetés----------------------------------------------
\subfile{sections/preamble}

%--Irodalomkutatás----------------------------------------
\newpage
\subfile{sections/literature}
%---------------------------------------------------------

%--Tartalmi rész------------------------------------------
	%--Alkalmazáshoz szükséges műszaki feltételek elemzése
	\newpage
	\documentclass[../documentation.tex]{subfiles}
 
\begin{document}
\section{Alkalmazáshoz szükséges műszaki feltételek elemzése}
\subsection{Projekt részletes leírása}
A projekt célja egy olyan robot demo hardveres és szoftveres kidolgozása, amely képes egy emberrel (továbbfejlesztés után akár egy másik robottal) lejátszani egy sakkjátszmát. A demo az ember-robot kollaboráció bemutatására szolgál, fontos szempont az interakció biztonságos megvalósítása mind az emberre, mind a környező tárgyakra tekintettel.\\

A megvalósításhoz a következő problémák megoldására van szükség:
\begin{enumerate}
	\item szükséges biztonsági funkciók beüzemelése,
	\item a bábuk helyzetének felismerése az egyes lépések előtt és után,
	\item a bábuk megfogása és mozgatása (ide tartozik a kalibráció és a referenciafelvétel),
	\item sakkalgoritmus beágyazása a programba,
	\item a sakkbábúk és a tábla megtervezése és megvalósítása,
	\item jelzés a robotkar számára, ha lépés történt.
\end{enumerate}
A felsorolt pontok a projekt során következőképpen kerültek kidolgozásra:
\begin{itemize}
	\item A bábuk helyzetének detektálása a projekt során QR-kód kereső és olvasó képfeldogozó eljárásokon alapul (a bábuk tetején található a kód). A kamera a roboton kerül rögzítésre.
	\item A bábuk mozgatása egy elektromosan vezérelt, párhuzamos megfogó (\angol{gripper}) segítségével történik.
	\item Ahhoz hogy a bábuk megfogása egyszerű legyen, azonos magasságú és azonos módszerrel megfogható bábuk készülnek.
	\item A biztonsági funkciók főként az tengelyekben ébredő plusz nyomatékok monitorozására és biztonsági zónák definiálására épül.
	\item A tábla és a robotkar, illetve a megfogó (egy jól definiált pontja) és a robotkar relatív helyzetének kalibrálására a robotvezérlő szoftverben elérhető alapfunkciók kerültek felhasználásra.
	\item A sakklépés megtörténtétét a robotvezérlőhöz kötött külső gombbal tudja a felhasználó jelezni.
	\item A képek fogadása, feldolgozása és a sakkalgoritmus futtatása mind a robotvezérlőn történik.
	\item Mivel a robotvezérlőn Java alapú környezet fut magas szinten, így a képfeldolgozó és a sakkozó programok is ebben lettek implementálva.
\end{itemize}

\subsection{A megfogó kialakításának és vezérlésének bemutatása}

\subsection{QR-kód generálás és képfeldolgozás}
A képfeldolgozáshoz lehetne saját mintát is használni a bábuk tetején, de a QR-kódok olvasására jól kidolgozott programok érhetőek el. A projekthez a nyílt forráskódú ZXing (``\angol{Zebra Crossing}'') program került felhasználásra\footnote{További információk és forráskód: https://github.com/zxing/zxing}. Ez a Java könyvtár (\angol{library}) alkamas különböző formátumú, egy- és kétdimenziós vonalkódokkal kapcsolatos képfeldolgozásra, ennek csak egyik eleme a QR-kód olvasás és generálás.

\subsubsection{A forráskód build-je}
A könyvtár egyszerű használatához célszerű .jar fájlokat generálni a forráskódból. Az első lépés a Github-on elérhető forráskód letöltése vagy klónozása a saját számítógépre. A programkód számos mappába és almappába van rendszerezve az egyes moduloknak megfelelően (pl.: core/ és javase/). \textbf{Fontos:} a mapparendszert olyan helyre tegyük a számítógépen, melynek elérési útjában nem található szóköz karakter (ékezetes karakter is probléma lehet)! Minden Java alapú modul esetén található egy pom.xml fájl, amit Apache Maven\footnote{Az Apache Maven program ingyenesen letölthető: https://maven.apache.org/} segítségével lehet használni.

Szükségünk van megfelelő java verzió telepítésére. A JRE (\angol{Java Runtime Environment}) helyett a JDK (\angol{Java Developement Kit}) valamelyik verzióját (a projekt jdk 10.0.2 verziót használ) érdemes telepíteni, ha fejlesztői funkciókat is igénybe szeretnénk venni (a JRE csomagot ez már tartalmazza). Az Apache Maven telepítése után szükségünk van az ehhez és a JDK-hoz tartozó környezeti változók beállítására. Ezt a Vezérlőpult->Rendszer->Speciális rendszerbeállítások->Környezeti változók...->Rendszerváltozók címszó alatt tehetjük meg. Szükségünk van egy `JAVA\_HOME' és egy `M2\_HOME' változóra (\ref{fig:envvar}. ábra).

Parancssorban navigáljunk a ZXing projekt gyökeréhez és futtassuk a `mvn install' parancsot a fordításhoz, a tesztekhez és az összes modul felépítéséhez. A ``-DskipTests'' paraméter hozzáadásával a unit teszteket kihagyhatjuk. Szükség lehet a `-Drat.ignoreErrors=true' paraméterre a licensz tesztekkel kapcsolatos problémák ignorálásához. A build folyamat akkor mondható sikeresnek, ha mindegyik modul felépítése sikeres (`ANDROID\_HOME' környezeti változó beállítása nélkül az Androidhoz kapcsolódó modulokat nem build-eli) (\ref{fig:buildsuccess}. ábra).

\begin{figure}[h]
\centering
\includegraphics{env-variable}
\caption{Szükséges környezeti változók beállítása}
\label{fig:envvar}
\end{figure}

\begin{figure}[h]
\centering
\includegraphics{zxingbuild}
\caption{Sikeres build végeredménye}
\label{fig:buildsuccess}
\end{figure}

A lefordított .jar fájlokat ezt követően az egyes modulokon belül találjuk. Például a lefordított \textit{core/} kód helye a \textit{core/target/core-x.y.z.jar}. Ezeket lehet beimportálni a képfeldolgozást megvalósító projektbe.

\subsubsection{ZXing könyvtár beimportálása és használata}
Az iiwa robotkart programozni Sunrise Workbench használatával a legegyszerűbb, ami egy JAVA Eclipse platformú szoftver. Emiatt praktikus okokból a szakdolgozat képfeldolgozási és sakkalgoritmus beágyazási része túlnyomó részt Eclipse-ben történt (verzió: 4.9.0)\footnote{Link: https://www.eclipse.org/}.

A könyvtár beimportálásához létrehoztam egy java projektet. Erre jobb egérgombbal kattintva elnavigáltam a `\angol{Configure Build Path...}' menüponthoz (\ref{fig:zxingimport}. ábra). A `\angol{Classpath}'-t kiválasztva jobb oldalt aktivizálódik az `\angol{Add External Jars...}' gomb. Kiválasztottam a \textit{core/} és a \textit{javase/} modulok .jar fájljait. Ezeken belül lehetőség van forráskód (\angol{Source attachment}) és dokumentáció (\angol{Javadoc location}) csatolására (\ref{fig:zxingimport}. ábra). `\angol{Apply and Close}' után megjelennek a csomagok a hivatkozott könyvtárak (\angol{Referenced libraries}) pont alatt.

\begin{figure}[h]
\centering
\includegraphics[scale=0.5]{buildpath}
\includegraphics[scale=0.5]{zxingimport}
\caption{ZXing hozzáadása a hivatkozott könyvtárakhoz}
\label{fig:zxingimport}
\end{figure}












\end{document}
	%--Robotkar kalibráció, referenciafelvétel kidolgozása
	\newpage
	\documentclass[../documentation.tex]{subfiles}
 
\begin{document}
\section{Robotkar kalibráció, referencia felvétel kidolgozása}
Ha a robotkarhoz szerszámokat szeretnénk csatolni vagy a mozgását a térben elhelyzkedő tárgyakhoz szeretnénk igazítani, akkor kalibrációs eljárásokat kell végrehajtani.

\subsection{Koordinátarendszerek}
Ha a robotkar pozíciójáról beszélünk, akkor elsősorban a végpontjának vagy az arra szerelt szerszámnak a pozíciójára és orientációjára vagyunk kiváncsiak. Az ilyen elhelyezkedést a térben különböző koordinátarendszerekkel és a közöttük felírható transzformációkkal tudjuk jellemezni. A főbb koordinátarendszereket foglalja össze \aref{tab:coordsystems} táblázat.

\begin{figure}[h]
	\centering
	\includegraphics[scale=0.40]{tcp-calibration}
	\caption{A TCP (\angol{Tool Center Point}) elhelyezkedése\cite{sunrisemanual}}
	\label{fig:tcp-calibration}
\end{figure}

Ahhoz hogy egy test pozícióját és orientációját definiálni tudjuk a térben legalább 6 lineárisan független koordinátára van szükség. A viszonyítási koordinátarendszerben felírt 3 transzlációs és 3 rotációs koordináta megfelel erre a célra.
\begin{itemize}
	\item Transzlációs vektorok:
	\begin{itemize}
		\item X távolság: a referencia KR X tengelye mentén vett transzláció
		\item Y távolság: a referencia KR Y tengelye mentén vett transzláció
		\item Z távolság: a referencia KR Z tengelye mentén vett transzláció
	\end{itemize}
	\item Rotációs vektorok:
	\begin{itemize}
		\item A szög: a referencia KR Z tengelye körül vett forgatás
		\item B szög: a referencia KR Y tengelye körül vett forgatás
		\item C szög: a referencia KR X tengelye körül vett forgatás
	\end{itemize}
\end{itemize}

\begin{table}[h]
\setlength\arrayrulewidth{1pt}
\begin{tabu} to\linewidth{| X[1.4,c,m] | X[3,l,m] |}
\hline
\rowfont{\bfseries\large} Koordinátarendszer (KR) & Leírás \\ \hline
Világ KR & A világ KR egy állandó, Descartes-i (Cartesian) koordinátarendszer. Ez minden más KR gyökere, mint például a bázis KR-é vagy a robot bázis KR-é. Alapértelmezetten a világ KR a robot bázispontjában található. \\ \hline
Robot bázis KR & A robot bázis KR-e olyan Descartes-i KR, amely a robotkar bázispontjában található. Ez definiálja a robotkar relatív helyzetét a világ KR-hez képest. Alapértelmezetten a robot bázis KR megegyezik a világ KR-rel. Meg lehet adni egy elforgatási vektort a Sunrise.Workbench-ben, amely definiálja a robot relatív forgatását a világ KR-hez képest. Alapbeállítésként a padlóra rögzített robot felszerelési orientációja: (A=0°, B=0°, C=0°).\\ \hline
Bázis KR & Ahhoz hogy a Descartes-i térben mozgásokat definiálhassunk szükség van referencia KR (bázis) felvételére. Sztenderd módon a világ KR a mozgás bázis KR-e. További bázis KR-eket lehet definiálni a világ KR-hez képest. Ezt mutatja be a FEJEZET HIVATKOZÁS!!!!. \\ \hline
\angol{Flange} KR & A flange KR írja le a robot \angol{flange} középpontjának a pozícióját és orientációját. Ennek az elhelyezkedése nem fix, a robottal együtt mozog. A \angol{flange} KR használható a rá fogatott szerszámokhoz kötődő KR-ek origójaként. Például a szakdolgozat keretében a gripper egy jól definiált pontjára illesztett KR a robot \angol{flange} KR-éhez kepést relatív lett megadva (\ref{fig:tcp-calibration}). \\ \hline
Szerszám KR & A szerszám KR az a Descartes-i KR, amely a felszerelt szerszám munkapontjára illeszkedik. Ezt hívják szerszám középpontnak (\angol{Tool Center Point} - \textbf{TCP}). Bármennyi frame-et lehet definiálni egy szerszámhoz és ezek mindegyike kiválasztható TCP-nek. A szerszám KR rendszerint a \angol{flange} KR-ből származtatott. \\ \hline
\end{tabu}
\caption{A főbb koordinátarendszerek}
\label{tab:coordsystems}
\end{table}

\begin{figure}
	\centering
	\includegraphics[scale=0.35]{tcp-example1}
	\includegraphics[scale=0.35]{tcp-example2}
	\caption{Szerszám 1 (balra) és 2 (jobbra) TCP-vel\cite{sunrisemanual}}
	\label{fig:tcp-examples}
\end{figure}

\newpage
\subsection{Szerszámkalibráció - XYZ 4-pont metódus}
A sakkprojekt kivitelezése során szükséges a robotkarra szerelt megfogó kalibrálása ahhoz, hogy a pozícionálás minél pontosabb legyen. A robotkar szoftverében ez egy alapfunkció, kiegészítő csomag nem szükséges hozzá. A kalibrációs eljárás alapvetően 2 lépésből áll:
\begin{enumerate}
	\item a TCP origójának meghatározásból,
	\item az origóra illesztett koordinátarendszer orientálásából.
\end{enumerate}
A második lépés jelen esetben elhagyható, megfelelő ha a TCP az orientációját a referencia KR-től örökli. 

A TCP origójának meghatározására használt módszer: XYZ 4-pont metódus. Ehhez ki kell választanunk a szerszám egy adott pontját adott helyzetben (pl.: gripper egyik csúcspontja nyitott állásban); ez lesz a TCP. Az eljárás lényege, hogy a kalibrálni kívánt szerszám ezen pontját egy referenciaponthoz vezéreljük 4 különböző irányból. A referencia pont szabadon választható. A robot kontroller a különböző \angol{flange} pozíciókból számolja ki a TCP helyzetét.

A 4 \angol{flange} pozíció egymás közötti távolságainak meg kell haladniuk egy előre definiált minimumot. Ha a pontok túl közel vannak egymáshoz, akkor a pozíció adatokat nem lehet elmenteni. Erre hibaüzenet figyelmeztet.

A kalibráció minősége a transzlációs vektor hibájával mérhető, amit a program kalibráció közben számít ki. Ha ez a hiba meghalad egy definiált határértéket, akkor célszerű a TCP-t újból kalibrálni.

A minimális távolságok és a maximális számítási hiba a \angol{Sunrise Workbench}-ben konfigurálható. Részletesebb leírás az eljárásról a megfelelő Sunrise.OS verzió manuáljában található\cite{sunrisemanual}.

\begin{figure}[h]
	\centering
	\includegraphics[scale=0.50]{tcp-calibration2}
	\caption{XYZ 4-pont metódus 4 lépése\cite{sunrisemanual}}
	\label{fig:tcp-examples}
\end{figure}

\subsection{Báziskalibráció - 3 pont metódus}
A báziskalibráció során a felhasználó egy Descartes-i koordinátarendszert (bázis koordinátarendszert) rendel a bázisnak választott frame-hez. A bázis KR-nek a felhasználó által választott, tetszőleges helyen lehet az origója.

A bázis kalibráció előnyei:
\begin{enumerate}
	\item A TCP végigvezérelhető a munkafelület szélein vagy egy munkadarabon.
	\item A bázishoz viszonyítva lehet felvenni a szükséges pontokat. Ez a szakdolgozat során azért fontos, mert így nem kell a sakktábla egyes mezőihez tartozó pontokat külön felvenni, a pozíciókat meg lehet adni a bázisponthoz képest, ami lehet például a sakktábla egyik sarokpontja.
\end{enumerate}

A 3-pont metódus során az origó és a bázis 2 további pontja kerül rögzítésre. Az origó felvétele után a kívánt X tengely pozitív felén egy tetszőleges pont rögzítése következik. Végezetül az XY sík első térnegyedébe eső, szintén tetszőlegesen választott pontot kell felvenni. Ez a 3 pont egyértelműen meghatározza a bázist.

A rögzített pontok origótól vett távolságának meg kell haladnia egy minimumot, illetve az egyenesek között is meg kell lennie egy minimum szögnek (origó - X tengely és origó - XY síkon felvett pont). Ha a pontok túl közel vannak vagy az említett szőgek túl kicsik, akkor a pozíció adatokat nem lehet elmenteni. Erre hibaüzenet figyelmeztet.

A minimális távolságok és szögek a \angol{Sunrise Workbench}-ben módosíthatók. Részletesebb leírás az eljárásról a megfelelő Sunrise.OS verzió manuáljában található\cite{sunrisemanual}.

\begin{figure}[h]
	\centering
	\includegraphics[scale=0.50]{basecalibration}
	\caption{Báziskalibráció 3-pont metódussal\cite{sunrisemanual}}
	\label{fig:basecalibration}
\end{figure}

\subsection{A szerszám tömegeloszlásának meghatározása}
A robotkaron tengelyeiben ébredő többlet nyomatékra meg lehet határozni egy maximális értéket, így ha a munkaterében lévő objektummal ütközik, detektálja és megáll. A robot saját tömegének dinamikus mozgatásához szükséges nyomatéka nem adódik hozzá ehhez a többlet nyomatékhoz. Adott szerszám robotkarra erősítésével viszont előfordulhat, hogy a maximális járulékos nyomatékérték átlépi valamelyik tengelyen a megengedett nyomatékértéket. Ezen hatás korrigálásához szükséges a szerszám tömegeloszlásának kalibrálása (a sakkbábuk, mint munkadarabok kalibrálása nem szükséges, azok tömege elhanyagolható).

A tömegeloszlás meghatározásához a robot először különböző méréseket futtat automatikusan a csuklótengelyek (utolsó három tengely: A5, A6, A7) mozgatásával. Ez alapján a robotkarra szerelt szerszám tömege és a tömegközéppontjának helye meghatározható. Másik lehetőség a szerszám tömegének megadása (például katalógus alapján), majd ez alapján a tömegközéppont helyének kalibrálása.

Az eljárás futtatásához nincs szükség plusz csomagra, a funkció a robotszoftver beépített eleme. A lépések a következők (automatikusan történik, a folyamat 1-2 percet vesz igénybe):
\begin{enumerate}
	\item A folyamat kezdetén az A7 tengelyt a vezérlő a 0 pozícióba mozgatja. Ezen kívül az A5 tengelyt úgy pozícionálja, hogy az A6-os tengelyre többletnyomatékot a súly ne fejtsen ki. Ekkor a szerszám tömegközéppontja abba a síkba esik, amit az A6 tengely és a gravitációs térerősségvektor kifeszít.
	\item A mérés futása során az A6 és az A7 tengelyek egy bizonyos tartományban vesznek fel helyzeteket:
	\begin{itemize}
		\item Alapbeállításként az A6 tengely -95° és +95° közötti vesz fel pozíciókat.
		Ha a munkatér nem elég nagy ehhez a mozgástartományhoz, akkor ez a tartomány csökkenthető.
		\item Az A7 tengely 0°és -90° között mozog.
	\end{itemize}
\end{enumerate}

A mérést befolyásoló tényezőkről információ a Sunrise kézikönyvében\cite{sunrisemanual} található.

\begin{figure}[h]
	\centering
	\includegraphics[scale=0.50]{loaddatacalibration}
	\caption{Az A6 tengely mozgástartománya: 1-es esetben teljes, 2-es esetben csökkentett tartomány\cite{sunrisemanual}}
	\label{fig:loaddatacalibration}
\end{figure}

















\end{document}
	%--Robotkar mozgásának definiálása és programozása
	\newpage
	\documentclass[../documentation.tex]{subfiles}
 
\begin{document}
\section{Robotkar mozgásának definiálása és programozása}
\subsection{Biztonsági beállítások}
Ahhoz hogy a robot működése során ne jelentsen veszélyt a munkaterében lévő emberekre és vagyontárgyakra, különböző biztonsági eszközök beszerelésére, illetve akitválására van szükség. A robotszoftverben implementált biztonsági funkciók között vannak módosítható és permanens elemek. Ezeket foglalja össze felsorolás jelleggel \aref{sec:safetyfunctions} fejezet. A továbbiakban csak a projekt során is aktív elemek kerülnek tárgyalásra.

\subsubsection{Vészleállító eszköz}
Az ipari robot esetében a vészleállító eszköz (\angol{EMERGENCY STOP	 device}) a smartPAD-en található vészleállító (\ref{fig:smartpad} ábra). Ezt a kapcsolót kockázatos szituációban vagy vészhelyzetben kötelező benyomni.

Ha az operátor benyomta a vészleállító gombot, akkor a robotkar \angol{Safety stop 1 (path-maintaining)} (leírás \aref{tab:terms} táblázatban) megállást hajt végre. Az EMERGENCY STOP eszközt el kell csavarni ahhoz, hogy a műveletek folytatódhassanak.

\subsubsection{Engedélyező eszköz}
A robotkar esetében az engedélyező eszközök (\angol{enabling devices}) a smartPAD-re szerelt engedélyező kapcsolók. 3 ilyen kapcsoló található a smartPAD-en (\ref{fig:smartpad} ábra). Ezek mindegyikének 3 állása van:
\begin{itemize}
	\item Nem behúzott
	\item Középső pozíció
	\item Teljesen behúzott (pánik pozíció)
\end{itemize}

A teszt módokban és CRR esetén (\ref{tab:terms}) a robotkar csak akkor mozgatható, ha legalább 1 engedélyező kapcsoló középső állásban van.
\begin{itemize}
	\item Az engedélyező kapcsoló elengedése \angol{Safety stop 1 (path-maintaining)} megállást fog okozni.
	\item A kapcsoló teljes behúzása is ugyanilyen megálláshoz vezet.
	\item 2 engedélyező kapcsolót a középső pozícióban tartani lehetséges néhány másodpercig. Ez lehetőséget ad arra, hogy az operátor fogást váltson. Ha 2 engedélyező kapcsoló párhuzamosan középső pozícióban van tartva 15 másodpercnél tovább, az szintén \angol{Safety stop 1 (path-maintaining)} megállást fog okozni.
\end{itemize}

Ha az engedélyező kapcsoló a középső állásban ragad, akkor a következő lehetőségek állnak rendelkezésre:
\begin{itemize}
	\item a kapcsoló teljes behúzása,
	\item az EMERGENCY STOP eszköz benyomása,
	\item illetve a Start gomb elengedése
\end{itemize}

\begin{figure}[h]
    \centering
    \begin{subfigure}[t]{0.5\textwidth}
        \centering
        \includegraphics[scale=0.4]{smartpad-front}
        \caption{smartPAD előlnézetből\\ 2: kulcsos kapcsoló\\ 3: EMERGENCY STOP eszköz \\ 9: Start gomb}
    \end{subfigure}%
    ~ 
    \begin{subfigure}[t]{0.5\textwidth}
        \centering
        \includegraphics[scale=0.4]{smartpad-rear}
        \caption{smartPAD hátulnézetből\\ 1, 3, 5: enegedélyező kapcsoló\\ 2: Start gomb}
    \end{subfigure}
    \caption{KUKA smartPad\cite{sunrisemanual}}
    \label{fig:smartpad}
\end{figure}

\subsubsection{Operációs mód rögzítése}
Ahhoz hogy a robot mozgása közben ne lehessen operációs módot váltani (pl.: T1 módból T2-be), a smartPAD-en található egy kulcsos kapcsoló (\ref{fig:smartpad} ábra). Ennek elfordítására a robot megáll, ekkor lehet tetszőleges operációs módot választani. Ezeket \aref{tab:terms} táblázat részletesen tartalmazza.

\subsubsection{Megállást előidéző események}
A sakkprojekt során az alábbi események megállást idézhetnek elő a program futása során (a csillagozott elemek permanensen a robotszoftverbe vannak programozva, ezeket módosítani nem lehet):
\begin{itemize}
	\item operációs mód váltás történt mozgás közben*,
	\item az engedélyező kapcsolót elengedték*,
	\item az engedélyező kapcsolót teljesen behúzták*,
	\item a smartPAD-en található \angol{EMERGENCY STOP} eszközt benyomták*,
	\item a tengelyekre előírt maximális többletnyomatékot valamelyik tengelyen meghaladja a terhelés.
\end{itemize}

\subsection{Mozgástípusok elméletben}
Ez a fejezet bemutatja a mozgások programozásának elméleti szabályait. A szakdolgozat során programozott elemeket \aref{sec:motionprogramming} fejezet tárgyalja.

\subsubsection{Mozgástípusok áttekintése}
Több egymás utáni mozgás végrehajtása során a mozgás kezdőpontja mindig az előző végpontja.
A következő mozgástípusokat lehet beprogramozni különálló mozdulatokként:
\begin{itemize}
	\item Point-to-point (PTP, Ponttól pontig)
	\item Lineáris mozgás (LIN)
	\item Körkörös mozgás (\angol{Circural motion}, CIRC)
	\item Manuális vezetés kézi vezető eszközzel
\end{itemize}

A következő mozgástípusokat lehet beprogramozni a CP (``\angol{Continuous Path}'' - ``folytonos út'') spline\footnote{A mozgatott pont spline mentén halad a térben} blokk\cite{sunrisemanual} részeként:
\begin{itemize}
	\item Lineáris mozgás (LIN)
	\item Körkörös mozgás (\angol{Circural motion}, CIRC)
	\item Polinomiális mozgás (SPL)
\end{itemize}

JP (``\angol{Joint Path}'' - ``tengely út'') spline\footnote{Mivel a robotkar 7 tengellyel rendelkezik, a mozgása közben a tengelyek állása nem egyértelmű. JP sline blokk kivitelezésénél a tengelyek szögsebessége folytonos, így egy kedvezőbb dinamikájú mozdulatsort visz véghez.} blokk\cite{sunrisemanual} programozásakor PTP elemeket lehet használni.

A LIN, CIRC, SPL, CP spline blokk mozdulatok CP (``\angol{Continuous Path}'') típusúak, míg a PTP és a JP spline blokk JP (``\angol{Joint Path}'' - ``tengely út'') típusúak.

\subsubsection{PTP mozgástípus}
A robot a TCP-t a leggyorsann úton vezeti el a végponthoz. A leggyorsabb út általában nem egyezik meg a térbeli legrövidebbel, tehát nem egy egyenes (\ref{fig:ptpmotion}). A görbe útvonalak lekövetése gyorsabb az egyenesnél, mivel a robot tengelyei forogva és szimultán mozognak.

PTP a gyors pozícionáló mozgás. A mozgás pontos útvonala nem előre megjósolható, de többszöri futtatásra is ugyanazt az útvonalat követi le, feltéve ha az általános feltételek nem változtak.

\begin{figure}[h]
    \centering
    \includegraphics[scale=0.4]{ptpmotion}
    \caption{PTP mozgás\cite{sunrisemanual}}
    \label{fig:ptpmotion}
\end{figure}

\subsubsection{LIN mozgástípus}
A robot a TCP-t egy térbeli egyenes mentén mozgatja a végponthoz. A LIN mozgás során a végpozíció konfigurációját a program nem veszi figyelembe.

\begin{figure}[h]
    \centering
    \includegraphics[scale=0.4]{linmotion}
    \caption{LIN mozgás\cite{sunrisemanual}}
    \label{fig:linmotion}
\end{figure}

\newpage
\subsection{A robotkar mozgatása} \label{sec:motionprogramming}
A projekt során a robotkarral egyszerű mozgásokat kellett végeztetni. Nem volt szükség bonyolultabb görbék és felületek lekövetésére, csak ponttól ponttig mozgásra.

A robot mozgása szekvenciálisan ismétlődik, ennek lépései a következők:
\begin{enumerate}
\setcounter{enumi}{-1} 
	\item a robotkar képfelvételi pozícióba vezérlése (a normál játékmenet közben mindig ebből a pozícióból indul, nem kell ide irányítani)
	\item a sakkprogram által meghatározott lépés kezdőmezője fölé irányítása
	\item csökkentett sebességgel a bábuért lenyúlás, majd a gripper záródása után szintén csökkentett sebességgel visszaemelés
	\item a lépés végpontja fölé vezérlés
	\item redukált sebességgel a bábu lerakása, majd a gripper kinyílása után szintén redukált sebességgel visszaemelkedés a bábu fölé
	\item képkészítési pozícióba való visszatérés
\end{enumerate}

A szakdolgozat keretében megírt program nem keresi meg automatikusan az ideális képkészítési pozíciót (továbbfejlesztési irány), ezt kézzel kell megtennünk. Erre a csuklók direkt vezérlése is alkalmas, de praktikusabb egy tetszőleges koordinátarendszerben XYZ tengelyek mentén mozgatni (például a sakktábla sarkán felvett bázis KR-ben).

Ahhoz hogy a kamera látóterébe az egész kalibrációs tábla beleessen a robotkarnak viszonylag magasra kell nyúlnia. Ezt a 7, illetve a 14 kilós iiwa robotkar mozgástere is megengedi. Kisebb méretű robottal a projekt megvalósításához nagyobb látószögű kamerára vagy több képkészítési pozícióra lenne szükség.

\begin{figure}[h]
    \centering
    \includegraphics[scale=0.4]{flowchart-robot}
    \caption{A robotvezérlés helye a folyamatban}
    \label{fig:flowchart-robot}
\end{figure}

Praktikus lehet egy olyan frame betanítása, ahol az orientáció olyan, ahogy a robotkarnak a bábut meg kéne fognia. Erről a frame-ről másolatot készítve annak X, Y és Z koordinátáját külön-külön be lehet állítani. A sakktábla bázispont és ez a frame alapján a tábla összes mezőjéhez lehet generálni egy alsó megfogó helyzetet (ahol a megfogó két pofája összezáródik) és egy felsőt, ahová a képkészítés után megy. Az alsó megfogó helyzet a gripper TCP-re vonatkozik, a robotkar mozgatásához célszerű ezt a pontot irányítani.

A hagyományos sakktábla 8x8-as, az alsó és felső pozíciót is beleszámolva ez 128 generált frame-et jelent. Ezek kiszámítása nem különösebben időigényes, de áttekinthetőbb a programkód, ha ezekből a fizikai pozíciókból a sakkprogramhoz hasonlóan egy (az alsó és felső helyzet miatt 2) 8x8-as tömböt építünk fel a sakkprogramhoz hasonló indexeléssel. Mivel a sakktábla méretei ismertek, ezt 2 egymásba ágyazott for ciklussal meg lehet oldani, a bázis koordinátarendszerhez a megfelelő offszetet hozzáadva.

A projekt során a megfogópofa egyik sarokpontja lett betanítva TCP-nek, így szükség van még egy plusz offszetre ahhoz, hogy a robotkar mozgatásakor a megfogó középvonala kerüljön a mező közepe fölé, ne pedig a TCP (\ref{fig:tcp-examples} ábra).

A bábuk mozgatására érdemes külön függvényt írni, ami fogadja a sakkprogram által visszaadott kezdő- és végpont indexeket, majd ez alapján az adott bábut áthelyezi a megfelelő mezőre. A főprogramban implementált `MovePiece' függvény ezt a célt szolgálja. A bábuk emelésekor és letételekor érdemes a PTP mozgatás helyett a LIN-t alkalmazni, mivel annak tetszőlegesen állítható a sebessége\footnote{A PTP esetén is meg lehet adni egy százalékos sebességredukciót a tengelyekre.}.










\end{document}
	%--Sakkalgoritmus beágyazása
	\newpage
	\documentclass[../documentation.tex]{subfiles}
 
\begin{document}
\section{Sakkalgoritmus beágyazása}
A robotprogramhoz felhasznált sakkalgoritmusok és főbb Java osztályok alapja egy versenyre készített, nyílt forráskódú sakk alkalmazás \cite{chessgui}. 2005-ben Arwid (Arvydas) Bancewicz első helyezést ért el ezzel az alkalmazásával az OBEA Számítógépes Programozó Versenyen (\angol{OBEA Computer Programming Contest}) 17 évesen. Az általa készített program rendelkezik grafikus felhasználói felülettel (továbbiakban GUI - \angol{Graphical User Interface}), a forráskód nagy része ennek megfelelő működtetéséhez lett megírva.

A szakdolgozat keretében megvalósított projektem során külön a sakk alkalmazáshoz felhasználói felületet létrehozni nem volt szükségszerű. A robotprogram továbbfejleszthető olyan formában, hogy folyamatosan kijelezze a jelenlegi sakkállást, így még inkább nyomon követhető a játék menete, így megkönnyítve a továbbfejlesztést. Ezen a felületen keresztül akár tetszőleges felállást lehetne konfigurálni a játék kezdetéhez.

A sakkprogram beágyazásának első fő kihívása a GUI elhagyásával egy konzolalkalmazás létrehozása. A program forráskódja viszonylag hosszú, sok osztály lett implementálva különböző csomagokba rendezve (12 csomag és ezeken belül 93 osztály). A program működésének megértéséhez jó kiindulási pont a mellékelt README fájl és a ``\angol{program documentation}'' mappában található szotver leírás (\angol{Software Documentation.doc})\cite{chessgui}. Habár ezek főként a GUI működését taglalják, az ehhez tartozó főprogram kódjából a meghívott függvények alapján ki lehet következtetni a működés struktúráját. Ezen felül a Java osztályoknak és az egyes csomagoknak beszédes nevük van, például a következő sakklépést az \angol{algorithm} (algoritmus) csomagban található különböző algoritmusokra alapozott osztályok metódusai határozzák meg.

A szakdolgozathoz felhasznált csomagok az alábbiak (ezeken belül is bizonyos funkciók ki lettek kapcsolva, de ezek jelentik a program magját):
\begin{itemize}
	\item chess.algorithm: Ebben a csomagban kerültek implementálásra a különböző sakkalgoritmusokat meghívó és végrehajtó utasítások. Ezen felül itt lettek implementálva az bábuk lehetséges lépései.
	\item chess.core: Ez a csomag tartalmazza a sakkhoz szorosan kötődő osztályokat, metódusokat (pl.: bábuk elhelyezkedése a táblán).
	\item chess.properties: Itt találhatóak a programozott sakkjáték szempontjából praktikus funkciók (pl.: játék jelenlegi állása, játékosok neve stb.).
\end{itemize}

\subsection{A chess.algorithm csomag - az algoritmusok működtetője}
A programban az alábbi sakkalgoritmusok lettek implementálva:
\begin{itemize}
	\item Alfa-béta vágás (AlphaBeta.java)
	\item Minimax elv (MiniMax.java)
	\item NegaScout algoritmus (NegaScout.java)
	\item Principal variation search (PrincipalVariation.java)
	\item Kvázi véletlenszerűen választott szabályos lépés (RandomGen.java)
\end{itemize}

A sakkprogram alapbeállításként az Alfa-béta vágást használja. Az Alfa-béta vágás egy olyan kereső algoritmus, amely igyekszik csökkenteni a minimax algoritmus keresési fájában lévő kiértékelt elemek számát. Ezt a módszert gyakran alkalmazzák kétszemélyes játékok (pl.: amőba, sakk, go, stb.) esetén gépi játékos programozására. Egy adott lépés kiértékelését akkor szakítja meg teljesen, ha legalább egy válaszlépés bebizonyítja, hogy a lépés rosszabb, mint a korábban vizsgált lépés. Az ilyen lépések további vizsgálata felesleges. Ha egy hagyományos minimax fára alkalmazzák, akkor ugyanazt a lépést adja majd vissza, mint amit a minimax algoritmus adna, de kimetszi azokat az ágakat a fában, amik a kimenetet nem befolyásolják.\footnote{Szöveg forrása angol nyelven: https://en.wikipedia.org/wiki/Alpha\%E2\%80\%93beta\_pruning}

\begin{figure}[h]
\centering
\includegraphics[scale=0.55]{alphabeta}
\caption{Az alfa-béta vágás szemléltetése. A kiszürkített részfák vizsgálata felesleges (a lépések jobbról balra történő kiértékelésekor). A max és min szintek reprezentálják a játékos, illetve az ellenfelének a lépéseit. \protect\footnotemark}
\label{fig:alphabeta}
\end{figure}
\footnotetext{A kép forrása: https://upload.wikimedia.org/wikipedia/commons/thumb/9/91/\\AB\_pruning.svg/600px-AB\_pruning.svg.png}

Az algoritmus két változót értékel ki minden lépésben, alfát és bétát. Az alfa érték reprezentálja azt a minimum pontszámot, amely az u.n. `maximalizáló' játékos számára már biztosítva van, illetve a béta érték az a maximum pontszám, ami az u.n. `minimalizáló' játékos számára biztosított. Kezdetben alfa értéke negatív végtelen, béta pozitív végtelen, azaz mindkét játékos a saját legrosszabb lépésével indít. A fa rekurzív vizsgálata során folyamatosan változik az értékük a játékosok által garantáltan elérhető értékekre. Amint a béta érték az alfa alá csökken az azt jelenti, hogy ez az állás (ha mindkét játékos részéről a legjobb lépéseket tételezzük fel) nem állhat elő, így további vizsgálatuk felesleges.

Az algoritmusokat a sakk szabályaihoz és sajátosságaihoz a MoveAlgorithm illeszti. Ez az osztály az alábbi funkciókat tartalmazza (egyéb kiegészítő, teszteléshez megírt függvényeken felül):
\begin{itemize}
	\item Definiálja az egyes bábuk lehetséges lépéseit mezőkre lebontva, tehát az erre vonatkozó metódusok pontosan megadják, hogy az egyes bábuk melyik mezőkről melyikekre léphetnek. Példának okáért a gyalog esetén külön a fehér bábukra és külön a feketékre is meg van határozva (getPawnMoves(Coord c) függvény), hogy a kiindulási pontjukról az ellenfél irányába egyet vagy kettőt is léphetnek. Bármely más esetben 1-et léphetnek előre (az ütéseket külön függvények definiálják, ezek elkülönítése a robotprogramozás esetén is kifejezetten előnyös).
	\item Az egyes bábuk lehetséges ütéseit külön függvények határozzák meg. A gyalog kivételével a többi bábunál ez megegyezik az előző pontban leírt lépésekkel.
	\item A bábuk játékhelyzettől függő összes lehetséges lépését, ütését a `getRealMoves', a `getRealAttacks' és a `getRealAll' függvények adják vissza. Ezek kizárján azon lépéseket, melynél az adott játékos királya a lépés előtt és utána is sakkban áll. Ezt úgy vizsgálja meg a program, hogy adott játékállásban ``meglépi'' a kívánt lépést, majd ellenőrzi, hogy a király sakkban maradt-e.
	\item Itt találhatóak még az egyes játékállások ``költségét'' meghatározó függvények, amelyek által visszaadott értékek az algoritmusok működtetésének alapjait jelentik.
\end{itemize}

\subsection{A chess.core csomag - a játék magja}
A chess.core csomag definiálja az algoritmusok működtetéséhez nem szorosan kapcsolódó osztályokat. A fő fájl a ChessGame.java, ennek segítségével lehet egy játékot elkezdeni. A példányosításakor az alábbi inicializáló lépések futnak le:

\begin{enumerate}
	\item A játék állapota inicializáltra változik (a játékállapotokról a chess.properties csomag kapcsán lesz bővebben szó).
	\item Alapértelmezetten az Alfa-béta kereső algoritmus kerül beállításra. Ezt a játék során bármikor lehet módosítani a `setAlgorithm' függvény segítségével.
	\item A játékosok neve alapértelmezetten ``Black'' és ``White'', ezt is bármikor lehet módosítani a játék közben. Kiindulásként a fehér játékos az ember, a fekete a robotkar, de a program viszonylag könnyedén átalakítható ha ezt meg szeretnénk cserélni.
	\item A hagyományos sakkjátszmák során a játékosok egy sakkórát (\ref{fig:chessclock} ábra) használnak arra, hogy maximalizálják a játék idejét. Mindkét játékosnak előre meghatározott és beállított ideje van a gondolkozásra. Ha az adott játékos lépett, lenyom az órán egy kapcsolót, melynek hatására a másik játékos ideje telik addig, amíg ő is vissza nem kapcsolja az órát. A program két virtuális órát használ erre a célra, melyek indítása és megállítása automatikusan történik minden lépésnél.
\begin{figure}[h]
\centering
\includegraphics[scale=0.40]{chessclock}
\caption{Hagyományos játékokhoz használt sakkóra - a piros zászló leesése jelzi, ha a játékos ideje elfogyott \protect\footnotemark}
\label{fig:chessclock}
\end{figure}
\footnotetext{A kép forrása: https://www.polishchess.com/garde-analog-chess-clock-classic-p-161.html}
	\item A tábla és az alap felállás inicializálása a Board osztály példányosításával történik. Ez az osztály tartja számon, hogy melyik mezőn milyen bábu található, emellett praktikus okokból a két király jelenlegi helyzetét külön változókban tárolja.
\end{enumerate}

A bábuk pozíciója kapcsán külön érdemes kiemelni, hogy a pozíciót a bábuk koordinátája határozza meg. A lehetséges koordinátákat egy 8x8-as tömb jelképezi. A tömb [0,0] indexű eleme a tábla A1 mezője, a [7,7] pedig a H8 (\ref{fig:chessboard} ábra).

\begin{figure}[h]
\centering
\includegraphics[scale=0.45]{chessboard}
\caption{A sakktábla mezőihez koordináták rendelése}
\label{fig:chessboard}
\end{figure}

Létrehozni lépést a Move osztály példányosításával lehet. A konstruktor egy kezdő és egy végponti $x$ és $y$ koordinátát vár, amelyek a bábu kiindulási- és végkoordinátái. Azt, hogy egy lépés szabályos volt-e úgy lehet ellenőrizni, hogy meghívjuk a sakkjáték példányunk `checkIfLegalMove' függvényét paraméterként átadva a lépést. Fontos, hogy a jelenlegi sakkjáték példányunkra hívjuk meg ezt a metódust, így biztos a pillanatnyi állás alapján dönt.

A lépések végrehajtása a szintén a ChessGame példányon belül implementált `movePiece' függvénnyel lehet. Ez első körben értelemszerűen megváltoztatja a bábuk helyzetét. Második lépésként az adott lépést hozzáadja a lépésekről vezetett listához (a ChessTableModel osztály segítségével). Ezek után a program átállítja, hogy ki a soron következő játékos, elindítja az óráját és ellenőrzi, hogy véget ért-e a játék, azaz mattot adott-e valamelyik fél.


\subsection{A chess.properties csomag - kiegészítő funkciók}
Ezen a csomagon tartalmaz számos a hagyományos sakkjáték végigjátszásához elengedhetetlen és néhány extra funkciót (a nélkülözhetetlen elemek ki lettek emelve a felsorolásban):

\begin{itemize}
	\item BoardParameters: továbbfejlesztés részeként a smartPAD-en meg lehetne jeleníteni a jelenlegi állást. Ehhez hasznos függvényeket és beállításokat tartalmaz ez az osztály.
	\item ChessColors: az előző pontban említett megjelenítéshez különböző színdefiníciókat bocsát rendelkezésre.
	\item ChessPreferences: segítségével el lehet menteni a jelenlegi játékhelyzetet. A program képes tetszőleges helyzetből kezdeni egy játékot, tehát ennek segítségével lehetséges egy játék későbbi folytatása.
	\item \textbf{GameParameters:} itt lehet beállítani az adott játékhoz használt sakkalgoritmus paramétereit (az algoritmus kiválasztása, kereső fa szintjeinek száma). Ezen felül itt lehet a játéknak címet adni, ez tárolja a játékosok nevét és azt, hogy melyik játékos helyett játszik a gép.
	\item \textbf{State:} meghatározza a játék jelenlegi állását (Vége van-e? Meg lett-e állítva? Inicializálva lett-e már? stb.).
\end{itemize}

\subsection{A sakkprogram kiegészítése a projekthez}
Ahogy \aref{fig:flowchart-chess} ábrán is látszik, a projekt megvalósításához a sakkprogramot össze kellett kötni a képfeldolgozást végző résszel és a robotvezérlővel. A képfeldolgozó rész meghatározza, hogy melyik mezőkön helyezkednek el a fehér bábuk, ez a sakkprogram bemenete. A kimenete pedig a szükséges lépés kezdő- és végpontja (pl.: [0,0] mezőről a [0,7] mezőre). Ezek fizikai koordinátákra transzformálása már a robotvezérlőn futó főprogramban történik. 

\begin{figure}[h]
\centering
\includegraphics[scale=0.45]{flowchart-chess}
\caption{A sakkprogram helye a folyamatban}
\label{fig:flowchart-chess}
\end{figure}

A képfeldolgozás kimenete egy 8x8-as tömb (mátrix), amely indexelése megegyezik a sakkprogramnál bemutatott mezőindexeléssel (\ref{fig:chessboard} ábra). Ez a tömb az ember lépése utáni állapotban mutatja a bábuk elhelyezkedését. Ahhoz, hogy a megtett lépést meg lehessen határozni, ezt az állapotot össze kell vetni a lépés előttivel. A programben vezetett állást bármikor le lehet kérdezni (sakkjáték példány -> board -> b). A kapott elem egy 8x8-as tömb, amely egyes elemei vagy null értékűek vagy az ott található bábut írják le. Minden bábu esetén le lehet kérdezni, hogy fekete-e vagy fehér (white: bábu \angol{boolean} tulajdonsága - igaz ha fehér, hamis ha fekete). Ezek alapján létre lehet hozni egy olyan logikai értékeket tartalmazó tömbböt, amely formailag megegyezik a képfeldogozó rész által szolgáltatott tömbbel.

A lépés előtti és utáni tömböket a `FindMove' függvény értékeli ki. Elemenként hasonlítja össze a tömböket a program. Ha két elem megegyezik, az azt jelenti, hogy az ott lévő bábu nem mozdult el. Ha különbözik a két érték, akkor két eshetőség állhat fenn:

\begin{itemize}
	\item a bábu ellépett onnan (lépés után `false' a mező értéke) - ekkor ez a mezőkoordináta lesz a lépés kiindulópontja, vagy
	\item a bábu oda lépett (lépés után `true' a mező értéke) - ez a mező a lépés végpontja.
\end{itemize}

Két fehér bábu egy kör alatt csak sánszoláskor mozdulhat el. Ezt az eshetőséget a lépés meghatározásakor külön meg kell vizsgálni. Mivel a fekete bábuk pozícióját a sakkprogram követi, így nem igényel külön eljárás kialakítását az, ha a fehér játékos leüti a fekete egy bábuját. Ilyen esetben a leütőtt bábu pozíciójához a lépés előtt `false' érték tartozott, ezt a sakkprogram adja meg. A képfeldolgozó rész csak a fehér bábukat keresi (azokon van csak zöld jelölő), így az ütés után a mező értéke `true' lesz.

A lépés validálására érdemes meghívni a sakkjáték `checkIfLegalMove' függvényét mielőtt a lépést a programban végrehajtatnánk, mivel szabálytalan lépést is be lehetne vinni.







\end{document}
	%--Eredmények értékelése
	\newpage
	\documentclass[../documentation.tex]{subfiles}
 
\begin{document}
\section{Eredmények, továbbfejlesztési irányok} \label{sec:results}
A projekt 4 főbb modulból áll (képkészítés, képfeldolgozás, sakkprogram és robotvezérlés), ezeket kellett összekötni ahhoz, hogy sakkozásra képessé lehessen tenni a robotkart \aref{sec:projectdescription} fejezetben bemutatott konstrukcióval. A projekt összesített értékelését \aref{sec:osszefoglalas}. fejezet tartalmazza. Ebben a fejezetben az egyes modulok korlátai, kiforrottsága, megbízhatósága és továbbfejleszthetősége kerül taglalásra.

\subsection{Képkészítés értékelése}
A robotvezérlő (KRC4) (és a rajta futó szoftver) kompatibilis a választott Logitech C270 HD webkamerával. A kamerát egyszerűen a robotvezérlőn található USB bemenetek valamelyikére lehet kötni. A robotkar és a robotvezérlő több méteres távolsága miatt USB hosszabító kábelre van szükség, 2 méteres passzív toldókábellel kommunikációs hiba nélkül üzemel a kamera.

A kamera kezelő osztály (CameraHandler.java) HD képes webkamera kezelésére lett kialakítva. Minden bizonnyal egyéb képkészítésre alkalmas eszközt is tudna kezelni, de ez további teszteket igényelne. A kamera megnyitási ideje erősen változó lehet a tipus függvényében: a Logitech kamera esetén ez pár másodperc, míg a Microsoft kameránál fél vagy 1 perc is lehetett. A Windows alapfunkciójaként elérhető kamera alkalmazás mindkettő kamerát 1-2 másodperc alatt megnyitotta, szóval az OpenCV kamerakezelésében lehet különbség.

Továbbfejleszthető ez a modul:
\begin{itemize}
	\item a felbontás és képarány automatikus választásával,
	\item több csatlakoztatott kamera esetén a megfelelő kiválasztásával,
	\item illetve a torzítások ellensúlyozásával.
\end{itemize}

\subsection{Képfeldolgozás megbízhatósága}
A képfeldolgozás alapvetően 2 elemből áll: a mezőkalibrációból és a fehér bábuk kereséséből.

A mezőkalibráció főként a kalibrációs sakktáblamintán található sarokpontok kereséséből áll. Ezek alapján történik a mezőkről (pontosabban a bábuk tetejéről) készült képek kivágása. Az OpenCV 2.4.13.6-os verziónál a 3.4.4 megbízhatóbban működik, ugyanazon a képen megtalál olyan sarokpontokat, amiket a régebbi verzó nem. A sarokpontok megtalálása nem csak a kalibrációs tábláról alkotott képrészlet minőségétól függ, hanem a képen található egyéb mintázatoktól is, főként ha annak vonalai kontrasztosak. A kalibrációs képet utólagosan lehet módosítani, hogy a sakktáblán kívüli részt adott színnel ki legyen töltve. Ezt manuálisan kell megtenni.

A fehér bábuk keresése a mezőkről kapott képeken belüli színszűrésen alapszik. A módszer RGB színskálát használ a zöld színű részek megtalálására, ami átlagosnak mondható fényviszonyok között kifejezetten jól működik. Rosszabb fényviszonyok között nem lett tesztelve a módszer, de ilyen esetben a HSV színskála használata (az RGB helyett) előnyösebb lehet. A tesztelt fényviszonyokkal a képek körülbelül 50\%-át tölti ki zöld szín, ha ott fehér bábu található (a tetejükre zöld lap van ragasztva). Ez alapján a bábuk pozíciójának kiértékelése egyszerű.

Továbbfejleszthetési irányok:
\begin{itemize}
	\item A projekt kidolgozásának idején az OpenCV egy újabb verzióját is kiadták, érdemes lehet azt beépíteni a programba.
	\item A kamerapozíciónálás során a megtalált sarokpontok szinekkel megjelölése (kvázi valós időben) elősegítené a megfelelő pozíció megtalálását, illetve a megfelelő pozíció megtalálását automatizálni is lehetne.
	\item Az \ref{qrsection} fejezetben ismertetett QR kód alapú eljárás kidolgozása további funkciókat tenne elérhetővé.
\end{itemize}

\subsection{A sakkprogram értékelése}
A sakkprogram elsődleges forrása egy kész, Java alapű, felhasználói interfésszel rendelkező program volt, ebből lettek a projekthez szükséges elemek kiemelve. További függvényekkel kellett kiegészíteni ezt a programorészletet ahhoz, hogy a képfeldolgozó és a robotvezérlési modulokkal kompatibilis legyen.

Az sakkprogram magja fel van készítve arra, hogy egy teljes sakkjátszámát le lehessen játszani, viszont a képfeldolgozási résszel összekötés olyan módon lett implementálva, amely a sáncolást nem tudja ilyen formában felismerni és kezelni. Ez a probléma pár függvénykibővítéssel megoldható.

A sakkprogram által használt algoritmus tetszőlegesen módosítható lenne játék közben is, de ennek kezelésére nem készült külön felhasználói bemenet. Jelenleg a sakkprogram kódjának megváltoztatásával lehet ezt állítani.

A program beépített virtuális sakkórát is használ, emellett a játékosok nevei is személyre szabhatóak (ezt a nevet írja ki a program bármely játékoshoz kötődő esemény kiíratásánál). A sakkprogram magja képes lenne adott játékállásból folytatni egy játékot, illetve elmenteni az adott állást.

Ezzel a modullal kapcsolatos legjelentősebb fejlesztési irány a smartPAD-re kidolgozott felhasználói interfész kifejlesztése, így lehetőség nyílna játék közben a különböző paraméterek állítására, illetve a folyamatok monitorozására.

\subsection{A robotvezérlés értékelése}
\subsubsection{A megfogó irányítása}
A robotvezérlő EtherCAT kommunikációt használó, digitális és analóg I/O modulok segítségével tudja vezérelni a megfogót. A jelenleg implementált funkciókhoz elég csak a digitális modulok használata, de az analóg modulokra is szükség van az összes gripperfunkció kihasználáshoz.

A pozícióvisszacsatolás segítene eldönteni, hogy a megfogónak ténylegesen sikerült-e megfognia a bábut. A szorítóerő beállítása jelenleg hardveresen történik, de ezt a programból is lehetne vezérelni.

\subsubsection{A robotkar vezérlése}
A robotkar kalibrálása a sakktáblához gyorsan kivitelezhető. A robot képes a képkészítési pozíció ismételt felvételére, illetve a sakkalgoritmus által meghatározott lépések kivitelezésére, leszámítva azokat a lépéseket, amikhez több bábú mozgatása is szükséges (ütések, sáncolás, en passant). Ezen lépések implementálása nem ütközik hardveres problémába, így viszonylag könnyen megvalósíthatóak.

Továbbfejleszthető elemek:
\begin{itemize}
	\item Ha a robotkar ütközés hatására megáll, akkor jelenleg csak manuálisan lehet visszatéríteni az eredeti program folytatásához. Ezt a visszavezetést a programból is meg lehetne oldani.
	\item A robotkar jelenleg a robot bázis koordinátarendszerben mozog, de praktikusabb lenne a sakktáblához rendelt bázist használni. Ekkor a folyamatok a sakktábla különböző irányú dőléseire sokkal kevésbé lennének érzékenyek.
	\item A sakktábla bázispontjának felvétele ``handguiding'' (kézi vezetés) használatával gyorsabb és kényelmesebb lenne.
\end{itemize}






















\end{document}
%---------------------------------------------------------

%--Összefoglalás------------------------------------------
%\include{sections/summary}
%---------------------------------------------------------

%--Források-----------------------------------------------
\addcontentsline{toc}{section}{Hivatkozások}
\bibliographystyle{unsrt}
%\bibliographystyle{acm}
\bibliography{references}
%---------------------------------------------------------

%--Angol összefoglalás------------------------------------
%\include{sections/summary_en}
%---------------------------------------------------------

\end{document}