%------------------------------------------
% Szakdolgozat
% 
% 
% Varga Roland
%
% Konzulens: Magyar László
%------------------------------------------

\documentclass[12pt,a4paper,twoside]{article}
\usepackage{dissert}
% \renewcommand{\familydefault}{ptm}

\begin{document}
\pagenumbering{roman}
\author{Varga Roland}
\title{Ember és robot interakciójának demonstrációja\\
		Sakkozó iiwa robotkar}
%--Szennycímoldal-----------------------------------------
\newpage\thispagestyle{empty}
\begin{center}
     VARGA ROLAND\\
     SZAKDOLGOZAT
\end{center}
%--Sorozatcímoldal----------------------------------------
%\newpage\null\thispagestyle{empty}
\newpage\thispagestyle{empty}
\begin{center}
     BUDAPESTI MŰSZAKI ÉS GAZDASÁGTUDOMÁNYI EGYETEM\\
     GÉPÉSZMÉRNÖKI KAR\\
     MECHATRONIKA, OPTIKA ÉS GÉPÉSZETI INFORMATIKA TANSZÉK\\[1ex]
     \resizebox{2.5cm}{!}{
          \includegraphics{figures/mogi.png}
     }\\[1ex]
     SZAKDOLGOZATOK
\end{center}
%--Címoldal-----------------------------------------------
%\newpage\null\thispagestyle{empty}
\newpage\thispagestyle{empty}
\begin{titlepage} %environment for unique titlepage design
\centering
\resizebox{4.5cm}{!}{
  \includegraphics{bme}
}\\[1ex]
{\bf BUDAPESTI MŰSZAKI ÉS GAZDASÁGTUDOMÁNYI EGYETEM}\\
{\bf GÉPÉSZMÉRNÖKI KAR}\\
{\bf MECHATRONIKA, OPTIKA ÉS GÉPÉSZETI INFORMATIKA TANSZÉK}\\[3cm]

{\LARGE \scshape Varga Roland}\\[2ex]
{\Large SZAKDOLGOZAT}\\[2ex]
{\Large \bf Ember és robot kooperációjának demonstrálása\\
		Sakkozó iiwa robotkar segítségével}\\[2ex]
{\itshape Demonstrating human-robot collaboration\\
				With chess-playing iiwa robotic arm}\\[5cm]

\begin{tabularx}{\textwidth}{XXXX}
Konzulens: & Témavezető: \\
\hspace{0.75cm} \itshape Magyar László & \hspace{0.75cm} \itshape Dr. Czmerk András \\
\hspace{0.75cm} tesztmérnök & \hspace{0.75cm} egyetemi adjunktus \\
\end{tabularx}\\[4cm]

{\Large Budapest, 2018}
\end{titlepage}

%--Záradék és nyilatkozatok-------------------------------
\documentclass[../documentation.tex]{subfiles}
 
\begin{document}
 \section*{Köszönetnyilvánítás}
 A szakdolgozat elkészítése során sokak segítségére számíthattam. Köszönöm a KUKA Hungária Kft.-nek, hogy eszközöket és színvonalas környezetet biztosított a projekt megvalósításához.Szeretném megköszönni a támogatást Magyar Lászlónak, konzulensemnek és munkatársamnak, akitől a legtöbb szakmai segítséget kaptam. Köszönet illeti a többi munkatársam is, akiket a szakdolgozat valamely eleme kapcsán felkerestem, mindegyikük rendkívül segítőkész volt.

Az általam tervezett sakkbábuk 3D kinyomtatásáért köszönettel tartozom a BME Polimertechnika Tanszéknek, ezen belül is Horváth Istvánnak és Dr. Kovács Norbertnek.

Köszönöm Dr. Czmerk Andrásnak, témavezetőmnek, hogy a szakdolgozat elkészítéséhez iránymutatást adott.

Ezentúl szeretném megköszönni a családomnak a belém vetett bizalmat, a törődést és a támogatást. Mindig számíthattam a segítségükre.

Köszönettel tartozom Pál Imrének, aki kiskoromban sakkozni tanított.

Végül de nem utolsó sorban szeretném megköszönni barátaimnak, hogy biztattak, és hogy segítettek a szakdolgozat és az egyetemi teendők mellett is jól érezni magam. Külön köszönettel tartozom a Martos Klubnak és a Drönk csapatának, hogy teret és lehetőséget biztosítottak a kikapcsolódáshoz.

\vspace{10mm}
\itshape{Budapest, 2018.12.05.}

\vspace{7mm}
\hspace{80mm} \itshape{Varga Roland}

\end{document}

%--Tartalomjegyzék----------------------------------------
\newpage\null\thispagestyle{empty}
\newpage\thispagestyle{plain}
\tableofcontents

%--Előszó-------------------------------------------------
%\documentclass[../documentation.tex]{subfiles}
 
\begin{document}
\section{Előszó}
Napjainkban egyre növekszik az igény arra, hogy a robotok az emberekkel kapcsolatba léphessenek, ne legyenek kerítésekkel elválasztva. Ilyen módon sokkal inkább ki tudják segíteni az embert, egyfajta technikai asszisztens szerepet láthatnak el. Olyan munkákat képesek elvégezni, amely az ember számára fárasztó, monoton vagy veszélyes lehet. 

A különféle robotokkal már nem csak ipari környezetben, hanem akár a gyógyászat területén is találkozhatunk. Ilyen robotok képesek például rehabilitációs tornákkal segíteni a betegeknek. Az ilyen mértékű együttműködésre tervezett robotok kialakításánál fontos szempont, hogy az ember biztonságban érezze magát mellettük. Ha egy robot tud vigyázni a körülötte lévőkre, mozgása gördülékeny, kiszámítható és jól alkalmazkodó, akkor sokkal könnyebb a bizalmat megadni neki.

A szakdolgozat célja az ember és a robot finom együttműködését bemutató robotalkalmazás kidolgozása. Ez egy társasjáték formájában valósul meg: lehetőség nyílik egy ipari robot ellen egy sakkjátszma lejátszására.


\end{document}
%--Jelölések jegyzéke-------------------------------------
%\newpage\null\thispagestyle{empty}
%\newpage\thispagestyle{plain}
%\section*{Jelölések jegyzéke}

%A táblázatban a többször előforduló jelölések magyar és angol nyelvű elnevezése, valamint a fizikai mennyiségek esetén annak mértékegysége található. Az egyes mennyiségek jelölése – ahol lehetséges – %megegyezik hazai és a nemzetközi szak-irodalomban elfogadott jelölésekkel. A ritkán alkalmazott jelölések magyarázata első előfordulási helyüknél található.
%--Footer-------------------------------------------------
%\newpage\null\thispagestyle{empty}
\newpage\thispagestyle{fancy}
%\include{boschfoot}

\pagenumbering{arabic}

%--Bevezetés----------------------------------------------
\subfile{sections/preamble}

%--Irodalomkutatás----------------------------------------
\clearpage
\subfile{sections/literature}
%---------------------------------------------------------

%--Tartalmi rész------------------------------------------
%--Robotkar mozgásának definiálása és programozása
	\clearpage
	\subfile{sections/projectdescription}
	%--A kamera kezelésével kapcsolatos csomag leírása
	\clearpage
	\subfile{sections/camerahandling}
	%--Alkalmazáshoz szükséges műszaki feltételek elemzése
	\clearpage
	\subfile{sections/imageprocessing}
	%--Sakkalgoritmus beágyazása
	\clearpage
	\subfile{sections/chess-algorithm}
	%--Robotkar kalibráció, referenciafelvétel kidolgozása
	\clearpage
	\subfile{sections/calibration}
	%--Robotkar mozgásának definiálása és programozása
	\clearpage
	\subfile{sections/motions}
	%--Robotkar mozgásának definiálása és programozása
	\clearpage
	\subfile{sections/grippercontrol}
	%--Eredmények értékelése
%	\clearpage
	\subfile{sections/results}
%---------------------------------------------------------

%--Összefoglalás------------------------------------------
\clearpage
\subfile{sections/summary-hu}
%---------------------------------------------------------

%--Források-----------------------------------------------
%\addcontentsline{toc}{section}{Hivatkozások}
\newpage
\bibliographystyle{unsrt}
%\bibliographystyle{acm}
\bibliography{references}
%---------------------------------------------------------

%--Angol összefoglalás------------------------------------
\clearpage
\subfile{sections/summary-en}
%---------------------------------------------------------

%--Rövidítések jegyzéke----------------------------
%\newpage
%\addcontentsline{toc}{section}{Rövidítések jegyzéke}
%\begin{abbreviations}
%	\item[HRC]	\angol{Human Robot Collaboration}
%	\item[IoT]	\angol{Internet of Things}
%	\item[IoE]	\angol{Internet of Everything}
%	\item[TCP]	\angol{Tool Center Point}
%	\item[HRI]	\angol{Human-Robot Interaction}
%\end{abbreviations}
%-------------------------------------------------------

%--Függelékek--------------------------------------
%\newpage
%\appendix
%\subfile{sections/appendix}

\end{document}